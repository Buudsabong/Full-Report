\chapter{บทนำ}
\label{chapter:introduction}

\section{ที่มาและความสำคัญ}
การสำรวจและตรวจสอบสายพันธ์ของนกโดยอาศัยเสียงของนกมาเป็นตัวบ่งบอกถึงความแตกต่างทางสายพันธุ์นั้น 
มีความสำคัญอย่างมากต่อจุดประสงค์ของการวิจัยทางด้านวิทยาศาสตร์ในหลายๆด้าน ซึ่งองค์ความรู้ที่ใช้ในการระบุเสียงของสายพันธุ์นกอย่างแม่นยำนั้น 
สามารถช่วยให้คนเราเข้าใจถึงการกระจายทางภูมิศาสตร์และเรื่องการของวิวัฒนาการของสายพันธุ์นกได้ด้วย และเรียกได้ว่าเป็นสิ่งสำคัญจำเป็นสำหรับการอนุรักษ์ความหลากหลายทาง
ชีวภาพที่มีให้คงอยู่อย่างยั่งยืน ซึ่งผู้ที่สามารถนำองค์ความรู้ตรงนี้ไปใช้ต่อยอดได้มีอยู่อย่างมากมาย ไม่ว่าจะเป็นผู้ที่มีความต้องการในการค้นคว้าเรื่องความหลากหลายทางชีวภาพ 
ผู้ที่มีความสนใจในความต่างของเสียงนกแต่ละสายพันธุ์ และแน่นอนนักปักษีวิทยาย่อมเป็นหนึ่งในผู้ที่สามารถใช้องค์ความรู้ในส่วนนี้ได้อย่างมีประสิทธิภาพมากที่สุด \par

การระบุสายพันธุ์ของนกโดยใช้การรวบรวมข้อมูลโดยการใช้เสียงของนกแต่ละสายพันธุ์มาระบุถือว่าเป็นวิธีที่ดี ง่าย และทำได้อย่างมีประสิทธิภาพมากกว่าการใช้รูปถ่ายของนกมาก~\cite{birdclef}
เนื่องจากว่าการถ่ายภาพนกเก็บไว้ตามสถานที่ต่างๆ ให้ข้อมูลที่เยอะและแม่นยำซึ่งถือว่าเป็นเรื่องที่ยากและท้าทายมาก และในการเก็บข้อมูลเสียงของนกนั้นมีความเป็นไปได้ที่สูงกว่าในการเก็บข้อมูลเสียงของนกที่มีความละเอียด 
ไม่เปลืองทรัพยากรในการเก็บ และได้รับข้อมูลมาอย่างครอบคลุมกว่าการถ่ายภาพนกเก็บไว้อย่างแน่นอน \par

การมีส่วนร่วมกันของโครงการทางวิทยาศาสตร์หลายๆโครงการ อย่างเช่น Xeno-Canto~\cite{xeno-canto} มหาวิทยาลัยเทคโนโลยีเชมนิทซ์ (Chemnitz University of Technology) และที่อื่นๆ
ได้มุ่งเน้นไปที่การบันทึกเสียงของนกไว้เป็นจำนวนขนาดใหญ่ เพื่อเพิ่มความเป็นไปได้โดยรวมทั้งหมดในการรับรู้สายพันธุ์ของนกผ่านการฟังเสียงของพวกมัน และเพื่อที่จะนำข้อมูลที่ได้ไปใช้ในการสร้างแบบจำลองเชิงลึก (Deep Learning Model) 
เพื่อทำให้กระบวนการในการรับรู้และจำแนกเสียงนกสามารถดำเนินการไปได้แบบอัตโนมัติ เช่นการสร้างแอพลิเคชันในการระบุเสียงนกที่ได้รับข้อมูลเข้ามา การคงไว้ซึ่งความหลากหลายทางชีวภาพ และการอนุรักษ์นกที่ใกล้สูญพันธ์ผ่านการ
ฟังเสียงและติดตามว่ายีงคงมีนกชนิดนั้นๆอยู่ \par

ความท้าทายในการจำแนกเสียงนกในอดีตได้รับข้อมูลเสียงนกในการเดินทางแบบทิศทางเดียว (Mono-directional recording) จากเครื่องมือที่ใช้ในการบันทึกเสียงในขณะนั้น~\cite{Kahl2019} และองค์ความรู้และระบบการระบุเสียงนกในครั้งนั้นสามารถระบุออกมาได้เป็นอย่างดีและได้ถูกนำไปพัฒนาต่อยอดเป็นแอพลิเคชันอย่างมากมายในปัจจุบัน 
ตัวอย่างเช่นแอพลิเคชันที่ชื่อ BirdGenie~\cite{birdgenie} ที่มีความสามารถในการบันทึกเสียงของนกในขณะนั้น และบอกมาว่านกชนิดนั้นคือนกสายพันธุ์อะไรซึ่งมีจำนวนนกที่บอกได้ทั้งหมดมากกว่า 100 สายพันธุ์ด้วย แต่อย่างไรก็ตามก็ยังคงมีการให้ความสนใจอย่างมากในการระบุสายพันธุ์ของนกจากข้อมูลเสียงนกแบบหลายทิศทาง (Omnidirectional)
ซึ่งสิ่งนี้จะช่วยให้มีการตรวจสอบสภาพแวดล้อมและเสียงของสิ่งรอบข้างได้อย่างแม่นยำมากขึ้น และข้อดีของวิธีการนี้คือจำนวนของความเป็นอคติที่มีอยู่ในการสุ่มตัวอย่าง (Sampling bias) จะน้อยว่าอคติที่มาจากการสำรวจและเก็บข้อมูลโดยนักวิทยาศาสตร์ทั่วๆไป อย่างไรก็ตามการรับรู้เสียงนกในสถาพแวดล้อมที่มีกิจกรรมจากเสียงรอบข้างอยู่ 
มีสัญญาณเสียงที่ซ้อนทับกัน และมีเสียงรบกวนในระดับสูงอาจทำให้มีความยุ่งยากในรับรู้และระบุสายพันธุ์นกได้ \par

จากที่กล่าวมาข้างต้นผู้วิจัยจะสร้างแบบจำลองในการคาดคะเนสายพันธ์ของนกแต่ละตัวขึ้นมาโดยใช้ลักษณะต่าง ๆ ที่สามารถสกัดมาได้จากข้อมูลเสียงของนก เพื่อนำมาเรียนรู้และสร้างแบบจำลองโครงข่ายประสาทเชิงลึกแบบคอนโวลูชัน (Deep Convolutional Neural Network) 
ซึ่งมีความเป็นไปได้ว่าจะให้ประสิทธิภาพในการคาดคะเนสายพันธุ์นกได้อย่างแม่นยำ 

\section{วัตถุประสงค์}
\begin{enumerate}
    \item เพื่อศึกษาโครงสร้างของแบบจำลองที่ใช้ในการจำแนกชนิดของนกด้วยไฟล์ที่ได้ทำการบันทึกเสียงและระบุสายพันธุ์ของนกไว้
    \item เพื่อจำแนกและระบุชนิดของนกจากไฟล์ที่บันทึกเสียงของนกไว้เพื่อให้ง่ายต่อการติดตามและเป็นประโยชน์ต่อปักษีวิทยา หรือเป็นประโยชน์ต่อนักชีววิทยาผู้ซึ่งมีความสนใจและค้นคว้าเกี่ยวกับเสียงของนก
    \item เพื่อใช้ในการปรับปรุงและพัฒนาระบบตรวจจับเสียงอัตโนมัติที่สามารถจดจำ และจำแนกเสียงของนกที่มาจากการบันทึกเสียงในรูปแบบที่หลากหลายได้
\end{enumerate}

\section{วิธีการดำเนินการ}
\begin{enumerate}
    \item วางแผนการดำเนินงาน
    \begin{itemize}
        \item กำหนดวัตถุประสงค์และขอบเขตของโครงงาน
        \item กำหนดแบบจำลองที่ใช้ในการทำนาย
        \subitem - แบบจำลองแบบอินเซปชัน (Inception model)
        \subitem - ImageNet~\cite{JiaDeng2009}
        \item กำหนดเครื่องมือที่ใช้ในการสร้างแบบจำลอง
    \end{itemize}
    \item ศึกษาเครื่องมือที่ใช้และทฤษฎีที่เกี่ยวข้อง
    \begin{itemize}
        \item ศึกษาทฤษฎีต่างๆที่เกี่ยวข้องกับการสร้างแบบจำลอง
        \subitem - ทฤษฎีการเรียนรู้เชิงลึก (Deep Learning) โครงข่ายประสาทเชิงลึก (Deep Neural Network) โครงข่ายประสาทแบบคอนโวลูชัน (Convolutional Neural Network)
        \subitem - โครงข่ายประสาทแบบซ้อนๆกัน (Recurrent neural network) หน่วยความจำระยะสั้นแบบยาว (Long Short-Term Memory)
        \item ศึกษาหลักการทำงานของเครื่องมือที่เลือกใช้
        \subitem - PyTorch~\cite{Pfeiffer1987}
        \subitem - Keras
        \subitem - Librosa~\cite{mcfee_librosa/librosa:_2020}
        \subitem - Sklearn
    \end{itemize}
    \item ดำเนินการจัดการข้อมูลและสร้างแบบจำลอง
    \begin{itemize}
        \item รวบรวมข้อมูลและจัดการข้อมูลให้อยู่ในรูปแบบที่เหมาะสม
        \item สำรวจและตรวจค้นข้อมูล (Data Exploration)
        \item ดำเนินการแยกคุณสมบัติที่สามารถหาได้จากข้อมูลที่เป็นเสียง (Feature Extraction)
    \end{itemize}
    \item ประเมินผลแบบจำลอง
\end{enumerate}

\section{ขอบเขตของงาน}
ขอบเขตของงานถูกแบ่งออกเป็น 3 ส่วนคือ ขอบเขตของข้อมูลที่ใช้ในการสร้างและฝึกแบบจำลอง ขอบเขตของข้อมูลที่ใช้สำหรับการประเมินประสิทธิภาพของแบบจำลอง 
และขอบเขตการวิจัยในการสร้างแบบจำลองโครงข่ายประสาทเชิงลึกแบบคอนโวลูชัน (Deep Convolutional Neural Network)

\begin{enumerate}
    \item ขอบเขตของข้อมูลที่นำมาใช้ในการสร้างและฝึกแบบจำลอง
    \begin{itemize}
        \item ชุดข้อมูลตัวเก่าที่ได้มาจากปี 2019 ที่ได้ถูกนำไปเพิ่มความสมบูรณ์ของชุดข้อมูล โดยมีเครือข่าย Xeno-canto~\cite{xeno-canto} เป็นผู้มีส่วนร่วมในการช่วยจัดหาชุดข้อมูลและเพิ่มส่วนขยายทางภูมิศาสตร์ภายในตัวในตัวชุดข้อมูลด้วย
        \subitem - ชุดข้อมูลที่ใช้ในการฝึกและสอนแบบจำลอง ประกอบไปด้วยเสียงบันทึกของนกชนิดต่าง ๆ จากทั้งอเมริกาเหนือ อเมริกาใต้และยุโรป โดยได้รับการสนับสนุนจาก Xeno-canto เป็นเสียงบันทึกที่มีคุณภาพสูงมากกว่า 70,000 รายการครอบคลุม 961 สายพันธุ์ 
        และภายในแต่ละเสียงบันทึกจะมีข้อมูลเกี่ยวกับตำแหน่งที่บันทึก วันที่และคำอธิบายอื่น ๆ ของผู้บันทึกด้วย
        \subitem - มีเมตาดาต้า (metadata) ประกอบคู่กันกับไฟล์เสียงของนกที่ได้มาด้วย ซึ่งภายในเมตาดาต้าจะประกอบด้วยข้อมูลคร่าวๆเกี่ยวกับการเก็บข้อมูลในครั้งนั้นๆมา เช่น สถานที่ที่ใช้ในการอัดเสียงนกที่ได้ วันที่ที่เก็บ และคุณภาพของเสียงที่ได้
    \end{itemize}
    \item ขอบเขตของข้อมูลที่ใช้สำหรับการประเมินประสิทธิภาพของแบบจำลอง
    \begin{itemize}
        \item ข้อมูลที่ใช้ในการตรวจสอบประสิทธิภาพและการทำงานของแบบจำลอง (Validation data)
        \subitem - ชุดข้อมูลที่ใช้ในการตรวจสอบประสิทธิภาพของแบบจำลอง ประกอบไปด้วยไฟล์เสียงที่บันทึกในประเทศเปรู และสหรัฐอเมริกาทั้งหมด 12 ไฟล์และแต่ละไฟล์จะมีรายละเอียดต่อวินาที (sample rate) อยู่ที่ 32 kHz และแต่ละไฟล์จะมีระยะเวลาถึง 10 นาที
        \item ข้อมูลที่ใช้ในการทดสอบแบบจำลอง
        \subitem - ชุดข้อมูลที่ใช้ในการทดสอบประกอบด้วยเสียงบันทึกจำนวน 153 ไฟล์ที่บันทึกในประเทศสหรัฐอเมริกาและเยอรมนี โดยแต่ละเสียงบันทึกจะมีระยะเวลา 10 นาทีและมีการซ้อนทับกันของเสียงนกในปริมาณมาก
    \end{itemize} 
    \item ขอบเขตการวิจัยในการสร้างแบบจำลองโครงข่ายประสาทเชิงลึกแบบคอนโวลูชัน
    \begin{itemize}
        \item ผู้สร้างแบบจำลองต้องการที่จะให้แบบจำลองที่สร้างขี้นมานั้น สามารถที่จะระบุสายพันธุ์ของนก ตามเสียงที่ได้มาอย่างถูกต้องและใกล้เคียงกับความเป็นจริงให้ได้มากที่สุด และด้วยความที่มีข้อมูลเสียงยกอยู่อย่างเป็นจำนวนมาก 
        จึงมีความจำเป็นที่จะต้องแยกคุณสมบัติของเสียงออกมาจาก ไฟล์เสียงที่ได้เป็นคุณสมบัติต่างๆไม่ว่าจะเป็นขนาดของเสียงตามระยะเวลาที่ได้ และการตัดภาพจากกราฟเสียงที่ได้ออกาเป็นส่วนเล็ก ๆ 
        ผ่านการใช้ Librosa~\cite{mcfee_librosa/librosa:_2020} เพื่อนำไปเรียนรู้บนแบบจำลองท่สร้างบนสถาปัตยกรรมต่างๆ ตามโครงข่ายประสาทเชิงลึกแบบคอนโวลูชันไม่ว่าจะเป็น ImageNet Inception  และอื่น ๆ โดยแบบจำลองจะถูกสร้างโดยใช้ 
        PyTorch และ Keras เพื่อให้ได้ผลลัพธ์ออกมาดีที่สุด และได้สร้างแบบจำลองออกมาเป็นจำนวนมาก เพื่อเปรียบเทียบประสิทธิภาพการทำงานของแต่ละแบบจำลอง และนำแบบจำลองที่ได้ผลลัพธ์ดีที่สุดมาใช้ต่อไป
    \end{itemize}
\end{enumerate}

\section{ประโยชน์ที่คาดว่าจะได้รับ}
\begin{enumerate}
    \item ระบบสามารถจดจำเสียงนกและแยกประเภทของนกได้อย่างถูกต้อง
    \item สามารถเลือกแบบจำลอง และสถาปัตยกรรมเพื่อนำมาใช้กับข้อมูลที่มีคามแตกต่างออกไปตามการใช้งานได้
    \item สามารถนำข้อมูลนกที่แยกประเภทจากระบบแล้วไปใช้ประโยชน์ในการติดตามและตรวจสอบสุขภาพของระบบนิเวศ หรือเป็นประโยชน์ในการศึกษาของนักนกวิทยา
    \item ระบบและแบบจำลองที่สร้างขึ้น สามารถนำไปแยกสายพันธุ์ของนกจากข้อมูลไฟล์เสียงบันทึกใหม่ๆ ที่ถูกนำเข้าไปใส่ไว้ในตัวระบบได้
\end{enumerate}

